\begin{tikzpicture}[function/.style={text centered, font=\ttfamily},
    arrow/.style={-Stealth, thick},
    ]

    % Definiendo la función
    \node[function] (def) at (0,0) {{\color{green!50!black}def}};
    \node[function] (name) at (1.2, 0) {nombre};
    \node[function] (params) at (3.7, 0) {(param1, param2)};
    \node[function] (puntos) at (5.6, 0) {:};

    
    % Cuerpo de la función
    \node[function] (docstring) at (3.1, -0.6) {\color{orange} '''Info de la función'''};

    \node[function, align=center] (commands) at (1.4, -2) {comando1 \\ comando2 \\ . \\ . \\ .};

    % Cuerpo de la función
    \node[function, below=3cm of name] (return) {\color{green!50!black} return};
    \node[function] (value) at (2.6, -3.5) {valor};


    % Explicaciones
    \node[function, left=1cm of def, align=center, font=\footnotesize] (defexp) {
        Palabra clave \\ 
        para definir \\ 
        una función
        };

    \node[function, above=1cm of name, font=\footnotesize] (namexp) {Identificador};

    \node[function, above=2cm of params, font=\footnotesize] (parexp) {Lista de parámetros (opcional)};

    \node[function, right=1.5cm of commands, align=center, font=\footnotesize] (comexp) {
        Cuerpo de la función, \\ 
        bloque de instrucciones
        };

    \node[function, right=1.5cm of puntos, font=\footnotesize] (punexp) {Dos puntos};

    % Valor de retorno
    \node[function, below= 0.5cm of return, font=\footnotesize] (rexp) {Palabra clave para devolver un valor (opcional)};
    
    %Indent
    \node[function, left=1.5cm of commands, align=center, font=\footnotesize] (indexp) {
        El cuerpo de la función \\
        debe tener sangría
        };

    % Flechas de conexión
    \draw[arrow] (return) -- (rexp);
    \draw[arrow] (def) -- (defexp);
    \draw[arrow] (name) -- (namexp);
    \draw[arrow] (params) -- (parexp);
    \draw[arrow] (puntos) -- (punexp);
    \draw[arrow] (commands) -- (comexp);
    \draw[arrow] (commands) -- (indexp);

\end{tikzpicture}
